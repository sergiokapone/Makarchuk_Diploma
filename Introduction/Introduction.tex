\chapter*{ВСТУП}% Тема

\addcontentsline{toc}{chapter}{ВСТУП}% Додаємо його в зміст

У даний момент у ракетній техніці наявні актуальні проблеми, зокрема такі, що заважають подальшому розвитку ракетобудування в цілому~\cite{Alemasov}. Однією з таких проблем є бар’єр питомого імпульсу: конкретні типи ракетних двигунів залежно від свого принципу роботи мають зумовлені ним обмеження ефективності. Деякі задачі ракетобудування сьогодні залишаються невирішеними саме через недостатню ефективність рушійних установок, пов'язану з обмеженнями енергообміну у схемі їх роботи.

Ефективність найпоширеніших ракетних двигунів --- хімічних (таких, що працюють за рахунок енергії згоряння паливної суміші за термодинамічним циклом ракетного двигуна), залежить у першу чергу від характеристик пального, а також від умов середовища на визначеній ділянці траєкторії ракети~\cite{Sutton}. Наразі існують лише симетричні рішення означеної проблеми -- знаходження оптимальних параметрів газодинамічного тракту двигуна, вибір найбільш високоенергетичних та масоефективних паливних компонентів та найбільш витривалих, а отже й дороговартісних матеріалів, що не вирішує проблему хімічних ракетних двигунів якісним чином.

Хімічні ракетні двигуни мають ряд обмежень характеристик, наприклад залежність питомого імпульсу від зовнішнього тиску. Окрім того, бар'єр питомого імпульсу для них полягає в тому, що енергія, що може бути надана робочому тілу рушійною установкою, обмежена значенням внутрішньої енергії паливних компонентів; можливість надання кінетичної енергії ззовні для підвищення швидкості витікання відсутня~\cite{Walther}.

В аналогічних типів ракетних двигунів (електричні, ядерні тощо) обмеження для надання енергії робочому тілу менш істотні порівняно з хімічними двигунами. Отже, асиметричним рішенням проблеми питомого імпульсу існуючих та найбільш поширених у ракетобудуванні рідинних двигунів може бути гібридна рушійна установка, що поєднує у собі параметри існуючих пристроїв, частково усуваючи бар'єр питомого імпульсу хімічного двигуна.

У цій роботі розглядається концепт вищеописаної гібридної електрохімічної рушійної установки літального апарата --- плазморідинного ракетного двигуна, що складається зі скомпонованих особливим чином рідинного ракетного двигуна і магнітоплазмодинамічного прискорювача. Така установка поєднує велику тягу РРД на номінальному режимі й високу швидкість витікання на режимі другої ступені за рахунок електродинамічного прискорення робочого тіла МГД-прискорювачем. Предметом дослідження є визначення оптимальної схеми компоновки двигуна, а також оцінка його основних параметрів.


Мета роботи –-- оцінка ефективності поєднання РРД і МПД-прискорювача у межах однієї силової установки літального апарата з точки зору процесів термо- і газодинаміки шляхом умовного з'єднання камери згоряння РРД і МПД-канала, моделювання термодинамічних процесів у такому двигуні, а також кількісної оцінки його параметрів.

Завдання роботи –-- описати оптимальну схему роботи плазморідинного ракетного двигуна; змоделювати термодинаміку камери згоряння і сопла РРД за присутності робочого тіла МПД-прискорювача; оцінити тягу та питомий імпульс досліджуваного двигуна, спираючись на параметри конкретного РРД та відповідної йому за геометричними та енергетичними параметрами МПД-установки; оцінити приріст ефективності у порівнянні з вихідним РРД. 
