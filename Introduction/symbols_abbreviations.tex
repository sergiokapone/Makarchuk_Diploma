\chapter*{ПЕРЕЛІК УМОВНИХ ПОЗНАЧЕНЬ}
\addcontentsline{toc}{chapter}{ПЕРЕЛІК УМОВНИХ ПОЗНАЧЕНЬ}% Додаємо його в зміст

{%

ЕРД~–--~електричний ракетний двигун

КЗ~–--~камера згоряння

ТД~--~розрахунок~–--~термодинамічний розрахунок

ККД~–--~коефіцієнт корисної дії

МПД-прискорювач~–--~магнітоплазмодинамічний прискорювач

ПРРД~–--~плазморідинний ракетний двигун

РД~–--~ракетний двигун

РРД~–--~рідинний ракетний двигун

РТ~–--~робоче тіло

ТНА~–--~турбонасосний агрегат

ЯРД~–--~ядерний ракетний двигун






%Латинські символи:

%$c$~---~теплоємність;

%$H$~---~ентальпія;

%$k$~---~кінетична енергія турбулентності;

%$M$~---~молярна маса;

%$\dot{m}_{k} $~---~середня швидкість реакції;

%$P$~---~абсолютний тиск;

%$p$~---~парціальний тиск;

%$R$~---~універсальна газова постійна;

%$T$~---~абсолютна температура;

%$u, v, w$~---~компоненти швидкості;

%$x, y, z$~---~координатні осі;


%$Y_i$~---~масова концентрація.



%Грецькі символи:

%$\alpha$~---~коефіцієнт надлишку повітря;

%$\lambda$~---~коефіцієнт теплопровідності, Вт/м·К;

%$\mu$~---~динамічна в'язкість, Н·с/м$^{2}$;


%$\rho$~---~густина, кг/м$^{3}$.




}














