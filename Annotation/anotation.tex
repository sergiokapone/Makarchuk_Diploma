% !TeX program = pdflatex
% !TeX encoding = utf8
% !TeX spellcheck = uk_UA
\restoregeometry
\begin{center}
	{РЕФЕРАТ}
\end{center}
%\totaltables~таблиць та

Пояснювальна записка дипломної роботи за обсягом становить   \pageref*{LastPage}~сторінки, містить \totaltables~таблиці та  \totalfigures~рисунків. Для дослідження було використано  \total{citenum}~бібліографічних найменувань.

%Робота присвячена питанню ефективного спалювання пилоподібного вугільного палива різного діаметру у суміші з газами, яка
%подається у проточний реактор. 

Метою роботи є доведення можливості поєднання РРД і МПД-прискорювача у межах однієї силової установки КА з точки зору процесів термо-, газодинаміки і фізики горіння шляхом умовного з'єднання камери згоряння РРД і МПД-канала і моделювання стабільної роботи двигуна у такій конфігурації. Предметом дослідження є аспекти роботи такої моделі, що визначають приріст питомого імпульсу двигуна, а саме вплив компоненти робочого тіла МПД-прискорювача на процеси, протікаючі у камері згоряння РРД.

Виконане числове моделювання потокiв робочого тiла рiдинного ракетного двигуна (РРД) заданих параметрiв (горюча сумiш газiв пiд високим тиском) i магнiтоплазмодинамiчного (МПД) прискорювача (iонiзуюча присадка металiчного дрiбнодисперсного калiю) та їх взаємодiї у межах одного робочого перерiзу. У ході аналізу результатів було з’ясовано, що профiлi температур i швидкостей дискретної фази (присадки) мають розподiл значень, близький до розподiлу параметрiв потоку РРД; досягається певна рiвновага мiж частинками й потоком, що впливає на результуючi робочi характеристики дослiджуваної схеми рушiйної установки.

Аналіз результатів проведеного моделювання показав, що МПД-компонент плазморідинного двигуна для компенсації малої витрати присадки за габаритів, не перевищуючих довжину камери досліджуваного РРД і маючи котушки технологічно досяжних параметрів має прискорювати потік струмами порядку 1...10 кА, що потребує підведення потужності порядку 2...20 МВт.

%\textbf{Мета.} Дослідити вплив розміру вугільних частинок на процес їх горіння у суміші метану та кисню.
%
%\textbf{Об’єкт дослідження}. Процеси теплообміну,  кінетика горіння, яка впливає на процес енерговиділення в  камері згоряння при подачі палива та  повітря.
%
%
%\textbf{Предмет дослідження}. Фактори які впливають на процес утворення вугільного недопалу та оксидів вуглецю при подачі палива та  повітря у камеру згоряння.
%
%\textbf{Завдання роботи.} Для досягнення мети необхідно:
%\begin{enumerate}[label=\alph*),ref=\alph*]
%	\item Для циліндричної модельної камери згоряння змоделювати процес горіння вугільного палива
%	\item Провести моделювання за допомогою програмного комплексу ANSYS Fluent
%	\item Знайти оптимальні розміри вугільних частинок
%	\item Знайти оптимальну емісію оксиду вуглецю
%	\item Знайти оптимальну концентрацію метану в суміші
%	\item Дослідити вплив висоти реактора на процес горіння
%\end{enumerate}
\vspace*{5mm}%
\textbf{Ключові слова:} \textit{числове моделювання}, \textit{теорія ракетних двигунів}, \textit{горіння}, \textit{газодинаміка}, \textit{дискретна фаза}, \textit{двигунобудування}, \textit{РРД}, \textit{МПД-прискорювач}, \textit{іонізуюча присадка}, \textit{питомий імпульс}.

\newpage

\selectlanguage{english}
\begin{center}
	SUMMARY
\end{center}
%\totaltables~table,

The diploma work explanatory note includes  \pageref*{LastPage}~pages of the text, \totaltables~tables and 
\totalfigures~illustrations. At the problem modern state analysis, overall  \total{citenum}~references were used.

The purpose of this work is to prove the possibility of combination of liquid propellant rocket engine (LPE) and magnetoplasmadynamic (MPD) acccelerator within a joint spacecraft propulsion system in terms of thermodynamics, fluid dynamics and combustion theory in a way of simulation of stable simultaneous work of LPE combustion chamber and MPD channel in a single volume. The object of this research is a number of aspects that determine an increase of specific impulse of the engine, namely the influence of MPD accelerator propellant on LPE chamber workflow conditions.

During the research a numerical simulation of combined LPE - MPD propellant fluid flow was provided, using the specifications of a particular engine and accelerator. During the results analysis it was found that the profiles of velocity magnitude and temperature of the discrete phase flow have a distribution of values similar to these of a LPE fluid flow; as a result, there appears a state of thermodynamic equilibrium between MPD propellant additive and LPE propellant flow; this would make a notable influence on parameters of a propulsion system described in the research.

Analysis of numerical simulation results has proved that, for a compensation of losses caused by low mass flow, MPD component of a described propulsion system, that should also have the acceptable size and field magnitude of the coils, has to accelerate the propellant by high currents (a value of 1...10 kA), that factor requires a remarkably high electric power input (2..20 MW).


\vspace*{5mm}%
\textbf{Key words:} \textit{numerical simulation}, \textit{rocket engine theory}, \textit{combustion}, \textit {fluid dynamics}, \textit {discrete phase}, \textit{spacecraft propulsion}, \textit{liquid propellant engine}, \textit{MPD accelerator}, \textit{ionizing additive}, \textit{specific impulse}.




