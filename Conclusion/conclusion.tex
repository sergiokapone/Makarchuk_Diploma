\chapter*{Висновки}

\addcontentsline{toc}{chapter}{ВИСНОВКИ}% Додаємо його в зміст

У даному дослідженні проведена оцінка приросту ефективності гібридної електрохімічної ракетної рушійної установки у порівнянні з сучасними рідинними двигунами, що використовуються у ракетах-носіях у даний момент.

Проаналізувавши існуючі конструкції турбоагрегатів РРД, зроблено висновок, що схеми відкритого, закритого (з допалюванням) циклів і циклу з фазовим переходом є малопридатними для використання у якості компонента плазморідинного двигуна.

Було проведено термодинамічний розрахунок дво- і трикомпонентних реакцій у камері плазморідинного двигуна за допомогою програмного комплексу відкритого доступу \texttt{Астра.4/рс}.

Було визначено, що проблема енергообміну між паливною системою РРД і МПД-прискорювачем у випадку живлення останнього від згенерованої потужності на валу турбіни є комплексною і ключовою у даній установці. Використання компактного джерела енергії мегаватної потужності дозволить підвищити тягу МПД-прискорювача, не знижуючи його питомий імпульс.

Схема з паралельною водневою турбіною для живлення МПД-установки і трикомпонентна схема, з конструктивних міркувань, були визначені як оптимальні для компонування з МПД-прискорювачем, оскільки містять додатковий турбоагрегат, що не зазнаватиме критичних навантажень і зумовлює дворежимність установки --- цей фактор може суттєво розширити сферу використання двигуна.

Описаний концепт рушійної установки літального апарата за належної оптимізації та застосування компактного джерела електроенергії для живлення МПД-прискорювача може бути використаний в якості двигуна верхньої ступені або розгінного блоку ракети-носія, в умовах, де необхідний передусім значний питомий імпульс. 
