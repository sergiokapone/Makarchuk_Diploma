%========================================================================================================
%
%										            Taskpage
%
%========================================================================================================


\newgeometry{
	showframe,
	footskip=0cm,%
	headsep=0cm,%
	top=1cm, %поле сверху
	bottom=1cm, %поле снизу
	left=2cm, %поле ліворуч
	right=1cm, %поле праворуч
}

\begin{spacing}{1.1}

\begin{center}{%
	\bfseries
	{Національний технічний університет України}\\
	{<<Київський політехнічний інститут імені Ігоря Сікорського>>}%
    }\\

	Навчально-науковий Фізико-технічний інститут\\
	Кафедра фізики енергетичних систем\\
\end{center}

\noindent Рівень вищої освіти --- другий (магістерський)\\
\makeatletter
\noindent Спеціальність  \@specialnist
\makeatother

\vspace*{1.5em}%
\noindent%
\hfill\begin{minipage}[t]{0.5\linewidth}
	<<ЗАТВЕРДЖЕНО>>\\
	Завідувач кафедри\\
	\mfield{0.2\linewidth}{}{}{}{(підпис)} \mfield{0.75\linewidth}{}{}{Монастирський~Г.~Є.}{(ініціали, прізвище)}\\
	\mfield{0.15\linewidth}{<<}{>>}{}{} \mfield{0.5\linewidth}{}{}{}{}~2023~р.\\
\end{minipage}

\begin{center}
	{\Large\bfseries ЗАВДАННЯ}\\
	{\bfseries на магістерську дисертацію студенту}\\
\end{center}
\makeatletter%
\noindent\mfield{0.8\textwidth}{}{}{Макарчуку Богдану Олексійовичу}{(прізвище, ім’я, по батькові)} \mfield{0.2\textwidth}{}{}{}{(підпис)}
\makeatother%

\begin{enumerate}[label*=\arabic*., labelindent=0pt, itemindent=0cm]
	\makeatletter%
	\item Тема роботи: \@title,

	\noindent\mfield{1\linewidth}{науковий керівник роботи}{}{\@kerivnyk}{(прізвище, ім’я, по батькові, науковий ступінь, вчене звання)},\vspace{-3mm}\\

	\noindent затверджені наказом по університету від \@univapproved
	\item Термін подання студентом роботи \@submission
	\item Об’єкт дослідження: \@searchobject
	\item Предмет дослідження: \@predmet
	\item Перелік завдань, які потрібно розробити: \@tasks
	\item Орієнтовний перелік ілюстративного матеріалу: \@illustrations

\end{enumerate}



\restoregeometry


\end{spacing}
%\afterpage{%
%	\newgeometry{<options>}
%	% material for this page
%	\clearpage
%	\restoregeometry
%}
